\documentclass{beamer}
\usepackage{multimedia}
\usepackage{graphicx}

\usetheme{Antibes}
\usecolortheme{beaver}
\hypersetup{colorlinks=true, urlcolor=red}


\title{Group 6: Enriched Traceroute}
\author{Mrinal Chandra Vinoth Kumar, Rafael Sanchez, Artur Shum}
\institute{Stevens Institute of Technology}
\date{6 October 2022}

\begin{document}

\frame{\titlepage}

\begin{frame}
\frametitle{Overview}

Traceroute typically spews out a poorly-formatted table of 
opaque raw data. We aim to build on top of the basic \texttt{tracert} algorithm,
make it look better, expand its functionality, and hopefully provide deeper insight
into the path of the packets as well as the health of the nodes along the path.

\end{frame}

\begin{frame}
\frametitle{Feature Ideas}
\begin{itemize}
  \item RTT Analysis
  \item Path Analysis
  \item Output Formatting Options
  \item Potential Optimizations
\end{itemize}
\end{frame}

\begin{frame}
\frametitle{RTT Analysis}

Similarly to the second assignment, we will provide insight regarding
the numerical distribution of roundtrip times such as standard deviation,
mean, as well as further numerical analyses as time allows. Additionally,
further control over the frequency, count of pings, TTL, and hop numbers
will be given to the user.

\end{frame}

\begin{frame}
\frametitle{Path Analysis}

Since the packets may take varied paths throughout the day, storing data regarding
the jumps to be compared later against previous runs will help analyze the health
of the nodes along the typical path to a destination. 

\end{frame}

\begin{frame}

\frametitle{Output Formatting Options}

To amend \texttt{tracert}'s poorly formatted output, output format options 
will be provided to lend the output data to further analysis at the user's choice.
Some variants of output will be \texttt{.csv/.tsv} files and the 
grouping of columns will be variable via an argument as well.

\end{frame}

\begin{frame}
  \frametitle{Potential Optimizations}

  Since \texttt{tracert} is a minimal program as is, it is unlikely we can make optimization the 
  focus of the project. However, if time allows and we find areas where it can be improved,
  we will also look to handle the route tracing task more efficiently.

\end{frame}

\begin{frame}
  \frametitle{Links}

  \begin{itemize}
  \item \href{http://youtu.be/Z2_QqqkHR54}{\underline{Prototype Traceroute Program with GUI Demo}}
  \item \href{http://github.com/mrinchanSIT/Traceroute}{\underline{\texttt{mrinchanSIT/Traceroute} on Github}}
  \end{itemize}
\end{frame}




\end{document}