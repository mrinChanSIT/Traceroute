\documentclass{beamer}

\usetheme{Antibes}
\usecolortheme{beaver}

\title{Group 6: Traceroute Optimization}
\author{Mrinal Chandra Vinoth Kumar, Rafael Sanchez, Artur Shum}
\institute{Stevens Institute of Technology}
\date{6 October 2022}

\begin{document}

\frame{\titlepage}

\begin{frame}
\frametitle{Overview}

Traceroute typically spews out results as a somewhat-formatted table of
raw data. We aim to build on top of the basic \texttt{tracert} algorithm
to expand its functionality and hopefully provide deeper insight
into the path of the packets.

\end{frame}

\begin{frame}
\frametitle{Feature Ideas}
\begin{itemize}
  \item RTT Analysis
  \item Path Analysis
  \item Output Formatting Options
\end{itemize}
\end{frame}

\begin{frame}
\frametitle{RTT Analysis}

As we did in the second assignment, we could provide insight regarding
the numerical distribution of roundtrip time such as standard deviation,
mean, as well as further numerical analyses as time allows.

\end{frame}

\begin{frame}
\frametitle{Path Analysis}

Since the packets may take varied paths throughout the day, another feature
idea is to schedule \texttt{tracert} calls at given intervals of the day,
providing a succinct view of the difference in paths as well as any
nodes previously seen that were noticed to have gone offline.

\end{frame}

\begin{frame}

\frametitle{Output Formatting Options}

To amend \texttt{tracert}'s poorly formatted output, output format options 
could be provided to lend the output data to further analysis at the user's choice.
Some variants of output could be \texttt{.csv/.tsv} files and the 
grouping of columns could be varied with a flag as well.

\end{frame}

\begin{frame}
  \frametitle{Code Example}
\end{frame}


\end{document}